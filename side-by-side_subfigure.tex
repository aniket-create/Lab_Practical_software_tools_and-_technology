\documentclass{article}
\usepackage{graphicx}
\usepackage{subcaption}

\title{Side-by-Side Graphics using Subfigure}
\author{Aniket Gupta}
\date{\today}

\begin{document}

\maketitle

\section{Introduction}
This document demonstrates how to include side-by-side graphics using the subfigure concept in LaTeX. Below are two images displayed side-by-side using the \texttt{subfigure} package.

\section{Graphics}
\begin{figure}[h!]
    \centering
    \begin{subfigure}{0.45\textwidth}
        \centering
        \includegraphics[width=\linewidth]{piggy dog.jpeg}
        \caption{PIGGY DOG}
        \label{fig:first}
    \end{subfigure}
    \hfill
    \begin{subfigure}{0.45\textwidth}
        \centering
        \includegraphics[width=\linewidth]{tiger.jpeg}
        \caption{TIGER}
        \label{fig:second}
    \end{subfigure}
    \caption{Side-by-Side Images}
    \label{fig:side_by_side}
\end{figure}

\section{Conclusion}
Using the \texttt{subfigure} package, you can easily place graphics or figures side by side in a LaTeX document, which is useful for comparative analysis or presenting multiple related images together.

\end{document}
